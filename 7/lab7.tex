\documentclass[11pt,a4paper]{article}

\usepackage{polski}
\usepackage[utf8]{inputenc}
\usepackage{url}
\usepackage{enumerate}
\usepackage[pdftex,linkbordercolor={0 0.9 1}]{hyperref}
\usepackage{amsthm,amsmath,amsfonts,amssymb,mathrsfs}

\newtheorem{tw}{Twierdzenie}[section]
\newtheorem{stw}[tw]{Stwierdzenie}
\newtheorem{fakt}[tw]{Fakt}
\newtheorem{lemat}[tw]{Lemat}

\newtheorem{df}[tw]{Definicja}
\newtheorem{ex}[tw]{Przykład}
\newtheorem{uw}[tw]{Uwaga}
\newtheorem{wn}[tw]{Wniosek}
\newtheorem{zad}{Zadanie}

\DeclareMathOperator{\R}{\mathbb{R}}
\DeclareMathOperator{\Z}{\mathbb{Z}}
\DeclareMathOperator{\N}{\mathbb{N}}
\DeclareMathOperator{\Q}{\mathbb{Q}}

\usepackage{fancyhdr}
\pagestyle{fancy}
\fancyhf{}
\fancyfoot[R]{\textbf{\thepage}}
\fancyhead[L]{\small\sffamily \nouppercase{\leftmark}}
\renewcommand{\headrulewidth}{0.4pt}
\renewcommand{\footrulewidth}{0.4pt}

\title{Funkcje ciagłe i rózniczkowalne}
\author{mboike}
\date{\today}

\begin{document}
\maketitle


\tableofcontents

\section{Funkcje ciagłe}

\begin{df} 
(funkcja ciagła). Niech f : (a, b) $\rightarrow$ $\R$ , oraz niech  $x_{0} \in (a, b)$. Mówimy, ze funkcja f jest ciagła w punkcie $x_{0}$ wtedy i tylko wtedy, gdy:
\end{df}

\[\forall _{\epsilon>0} \exists_{\delta>0} \in (a, b) | x-x_{0} | < \delta \Rightarrow |f(x) - f(x_{0})| < \epsilon \]



\begin{ex}  Wielomiany, funkcje trygonometryczne, wykładnicze, logarytmiczne
sa ciagłe w kazdym punkcie swojej dziedziny.
\end{ex}

\begin{ex} Funkcja f dana wzorem:

\[f(x) = \left\{ \begin{array}{l} x+1 \hspace{10bp} dla \hspace{10bp} x \not= 0 \\ 0 \hspace{30bp} dla \hspace{10bp} x = 0 \end{array} \right. \]  
\\
Jest ciagla w kazdym punkcie poza $ x_{0} = 0 $.\\
Niech $\Q$ oznacza zbior wszystkich liczb wymiernych.

\end{ex}

\begin{ex} Funkcja f dana wzorem:

\[f(x) = \left\{ \begin{array}{l} 0 \hspace{10bp} dla \hspace{5bp} x \in \Q \\ 1 \hspace{10bp} dla \hspace{5bp} x 
\notin \Q \end{array} \right. \]
\\
nie jest ciagla w zadnym punkcie.
\end{ex}

\begin{ex} Funkcja f dana wzorem:

\[f(x) = \left\{ \begin{array}{l} 0 \hspace{10bp} dla \hspace{5bp} x \in \Q \\ x \hspace{10bp} dla \hspace{5bp} x
\notin \Q \end{array} \right. \]
\\
jest ciagla w punkcie $x_{0}$ = 0, ale nie jest ciagla w pozostalych punktach dziedziny.
\end{ex}

\begin{zad}
Udowodnij prawdziwosc podanych przykladow.
\end{zad} 

\begin{df}
Jesli funkcja f : A $\rightarrow \R$ jest ciagla w kazdym punkcie swojej dziedziny A to mowimy krotko, ze jest 
ciagla.
\begin{center}
Ponizsze twierdzenie zbiera podstawowe wlasnosci zbioru funkcji ciaglych.
\end{center}
\end{df}

\begin{tw}
Niech funkcje f,g: R $\rightarrow \R$ beda ciagle, oraz niech $\alpha,\beta \in \R$
Wtedy funkcje:
\\

a) $h_{1}(x) = \alpha * f(x) + \beta * g(x),$

b) $h_{2}$(x) = f(x) * g(x),

c) $h_{3}(x) = \frac{f(x)}{g(x)}$ (o ile g(x) $\not=$ 0 dla dowolnego x $\in \R$ ),

d) $h_{4}$(x) = f(g(x)), 
\\
\\
sa ciagle.
\\
\\
\indent
Nastepne twierdzenie zwane powszechnie "wlasnoscia Darboux" lub twierdzeniem
o wartosci posredniej ma liczne praktyczne zastosowania. Mowi ono o tym,
ze jesli funkcja ciagla przyjmuje jakies dwie wartosci, to przy odpowiednich zalozeniach
co do dziedziny, przyjmuje tez wszystkie wartosci posrednie. Mozemy sobie to
latwo wyobrazic na przykladzie funkcji, ktora opisuje zmiane temperatury w czasie.
Jesli o 7:00 bylo -1$^\circ$C, a o 9:00 bylo 2$^\circ$C, to zapewne gdzies miedzy 7:00 a 9:00 byl
taki moment, ze temperatura wynosila dokladnie 0$^\circ$C.
\end{tw}

\begin{tw}
Niech f: [a,b] $\rightarrow \R$ ciagla, oraz niech f(a) $\not=$ f(b). Wtedy dla dla dowolnego $y_{0} \in conv\{f(a), 
f(b)\}$

\end{tw}


\section{Rózniczkowalnosc}

\begin{df}
Niech f:(a,b) $\to \R, x_{0} \in$ (a,b) oraz f ciagla w otoczeniu punktu $x_{0}$. Jesli istnieje granica:  

\[lim_{x \to x_{0}} \frac{f(x)-f(x_{0})}{x-x_{0}}\]
\\
i jest skonczona, to oznaczmy ja przez f'(x$_{0}$) i nazywamy pochodna funkcji f w punkcie x$_{0}$.
\end{df}

\begin{df}
Jesli funkcja f posiada pochodna w kazdym punkcie swojej dziedziny,
to mowimy, ze f jest rozniczkowalna. Istnieje wtedy funkcja f', ktora kazdemu
punktowi z dziedziny funkcji f przyporzadkowuje wartosc pochodnej pochodnej
funkcji f w tym punkcie.
\end{df}

\begin{ex}
Wielomiany, funkcje trygonometryczne, wykladnicze, logarytmiczne
sa rozniczkowalne w kazdym punkcie dziedziny.
\end{ex}

\begin{ex}
Funkcja f(x) = |x| jest ciagla, ale nie posiada pochodnej w punkcie x$_{0}$ = 0.
\end{ex}

\begin{tw}
Niech f: [a,b] $\to \R$ ciagla i rozniczkowalna na (a,b). Dodatkowo niech f'(x) $\not=$ 0 dla x $\in$ (a,b), oraz 
niech m=min$_{x\in[a,b]}$f(x), M=max$_{x\in[a,b]}$f(x).
Wtedy na pewno f(a)=m, f(b)=M lub f(a)=M i f(b)=m. 

\end{tw}




\end{document}
